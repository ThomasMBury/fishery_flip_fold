
%----------------------------------------------------------------------------------------
%	PACKAGES AND OTHER DOCUMENT CONFIGURATIONS
%----------------------------------------------------------------------------------------

\documentclass[10pt]{article} % A4 paper and 11pt font size

\usepackage[T1]{fontenc} % Use 8-bit encoding that has 256 glyphs
\usepackage[english]{babel} % English language/hyphenation
\usepackage{amsmath,amsfonts,amssymb} % Math packages
\usepackage{enumitem} % Package for lists
\usepackage{graphicx} % Package for figures
\usepackage{changepage} % Allows for change of margin during paragraph
\usepackage[total={6.7in,9in}]{geometry} % Span of text over the page
\usepackage{verbatim}
\usepackage{float}
\usepackage{bm}		% For bold greek letters
\usepackage{graphicx}
\usepackage{caption}
\usepackage{subcaption}

\usepackage{sidecap}

% definition equals sign
\newcommand\defeq{\mathrel{\overset{\makebox[0pt]{\mbox{\normalfont\tiny\sffamily def}}}{=}}}



\usepackage{ntheorem}
\theoremstyle{break}
\newtheorem{theorem}{Theorem}[section]


\usepackage{fancyhdr} % Custom headers and footers
\pagestyle{headings} % Makes all pages in the document conform to the custom headers and footers

\setlength\parindent{0pt} % Removes all indentation from paragraphs

\renewcommand{\vec}[1]{\bm{#1}} % Use bold vectors

\DeclareMathOperator{\Tr}{Tr} % For trace and determinant text in equations
\DeclareMathOperator{\Det}{Det}
\DeclareMathOperator{\Var}{Var}
\DeclareMathOperator{\Cov}{Cov}






%----------------------------------------------------------------------------------------
%	TITLE SECTION
%----------------------------------------------------------------------------------------

\newcommand{\horrule}[1]{\rule{\linewidth}{#1}} % Create horizontal rule command with 1 argument of height

\title{	
\normalfont \normalsize 
\textit{The University of Waterloo} \\ [10pt] % Your university, school and/or department name(s)
\horrule{0.5pt} \\[0.4cm] % Thin top horizontal rule
\huge Discrete Population Model (Ricker, Harvesting) % The assignment title
\horrule{2pt} \\[0.5cm] % Thick bottom horizontal rule
}

\author{Tom Bury} % Your name

\date{\normalsize\today} % Today's date or a custom date

\begin{document}

\maketitle % Print the title


%\setcounter{tocdepth}{2}
%\tableofcontents





%-----------------------
% Model Framework
%------------------------


\section{Model Framework}

\begin{itemize}

\item Generic discrete population model subject to exploitation - used to model population dynamics of a variety of organisms (fisheries, insects, birds)- Ricker-type model.

\item Model reads
\begin{equation}
N_{t+1} = N_t e^{(r_t(1-N_t/K) + \sigma_E \epsilon_{E,t} )} - F\frac{N_t^2}{N_t^2 + h^2}
\end{equation}

\begin{itemize}
\item $N_t$ population biomass at time t
\item $r_t = r_0e^{\sigma_r \epsilon_{r,t}}$ intrinsic growth rate with demographic stochasticity (exponential filtering used)
\item $K$ carrying capacity of population - finite due to density dependent effects
\item $F$ maximum harvesting rate
\item $h$ half saturation of sigmoidal exploitation term
\item $\epsilon_{r,t}$, $\epsilon_{E,t}$ Gaussian noise terms for demographic and environmental stochasticity with mean zero standard deviations $\sigma_r$ and $\sigma_E$.


\end{itemize}

\end{itemize}





\begin{figure}
\centering
\includegraphics[width=\textwidth]{../figures/fish_ews.png}
\caption{Early warning signals preceding the Fold and Hopf bifurcation}
\end{figure}








\pagebreak

%----------------------
% Data analysis
%---------------------

\section{Data analysis}



















% ---------------------------
% Bibliography
%---------------------------

\pagebreak

\bibliographystyle{unsrt}
\bibliography{../../../bibliographies/critical_transitions.bib}








\end{document}


